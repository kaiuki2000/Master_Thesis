%%%%%%%%%%%%%%%%%%%%%%%%%%%%%%%%%%%%%%%%%%%%%%%%%%%%%%%%%%%%%%%%%%%%%%%%
%                                                                      %
%     File: Thesis_Conclusions.tex                                     %
%     Tex Master: Thesis.tex                                           %
%                                                                      %
%     Author: Andre C. Marta                                           %
%     Last modified :  4 Mar 2024                                      %
%                                                                      %
%%%%%%%%%%%%%%%%%%%%%%%%%%%%%%%%%%%%%%%%%%%%%%%%%%%%%%%%%%%%%%%%%%%%%%%%

\chapter{Conclusions}
\label{chapter:conclusions}

% I've re-read this once, and I think it's okay. I will, of course, have to read this again, many times, to make sure it's good (for submission).

We've introduced a new framework, inspired by \acrshort{qaoa} and \acrshort{qemc}, for seamlessly designing multiple distinct \acrshort{vqa}\textcolor{gray}{s}. This framework builds on \acrshort{qemc}'s core components while integrating concepts from \acrshort{qaoa} to leverage the strengths of both algorithms for improved performance. Tentatively named the \acrshort{iqaqe} Framework, it opens the door to numerous unexplored and unknown \acrshort{vqa}\textcolor{gray}{s}, offering a variety of parameters to experiment with. It can be seen as a testbed for developing new \acrshort{vqa}\textcolor{gray}{s}. While this may not be revolutionary in itself, it has the potential to pave the way for groundbreaking results in the future.


% ----------------------------------------------------------------------
\section{Main Findings}
\label{section:findings}

The primary achievement of this work is the development of the \acrshort{iqaqe} Framework itself. We believe this new framework has the potential to compete with current state-of-the-art algorithms, as we've demonstrated. It offers reduced qubit requirements compared to \acrshort{qaoa} and greater resilience to statistical uncertainty associated with small shot numbers, compared to \acrshort{qemc}, i.e., the results don't significantly deteriorate when we use finite shot numbers.

We also introduced various heuristics for selecting the number of qubits, the cardinality of lists, and mappings. These heuristics have generally yielded results that are competitive with \acrshort{qemc}. In certain scenarios, such as \acrshort{ndcnot} \acrshort{qemc}, we have surpassed the classical state-of-the-art (\acrshort{gw}) for a reasonably large graph ($32$ nodes). This promising result highlights the potential of the \acrshort{iqaqe} Framework. % to transform the field of \acrshort{vqa}\textcolor{gray}{s}.

Additionally, we proposed a scheme for determining the optimal mapping for a specific graph based on statistical properties of the mappings themselves, using a small machine learning model. Although the results were not entirely satisfactory, we believe this approach can inspire future research, building on this idea and our initial work. % The concept is sound, but the execution and model require further refinement.


% ----------------------------------------------------------------------
\section{Future Work}
\label{section:future}
There are several promising directions for future research building on the current work. Key areas of exploration include improvements to the machine learning (\acrshort{ml}) model, the development of new heuristics, and testing on larger and more complex graphs. Let us take a closer look at these.
\vspace{5mm}

\noindent\textbf{Enhancements to the Machine Learning Model}: % Further work on the \acrshort{ml} model is essential.
\noindent Exploring the use of neural networks could potentially yield better results. Investigating additional input variables and incorporating more data points will likely improve the model's accuracy. Another avenue worth exploring is the use of Support Vector Regression (SVR) as an alternative approach, for non-linear model fitting.
\vspace{5mm}

\noindent\textbf{Development of New Heuristics and Exploration of Different Schemes}: Developing new and creative heuristics is crucial. For example, testing different cardinalities for different nodes may reveal an optimal configuration that has not yet been identified. This exploration could uncover a "sweet spot" that enhances the performance of the algorithm. Moreover, in the context of \acrshort{qemc}'s encoding, experimenting with various methods to derive node probabilities from the basis states' probabilities could be interesting. Exploring alternatives to the current "sum and normalize" method may yield better results. Additionally, trying different (problem-inspired) ansätze and cost functions could lead to further improvements. For instance, developing a mapping-inspired ansatz that depends on the mapping itself may offer novel insights and enhance performance. Finally, expanding on the alternative schemes (Chapter \ref{chapter:Exploratory_Ideas}) mentioned in this work could provide valuable additional avenues of research.
\vspace{5mm}

\noindent\textbf{Testing on Larger Graphs and Scalability Analysis}: Expanding the testing of the framework to larger graphs and a greater number of instances will enhance the validation of its effectiveness. Conducting more extensive and rigorous testing will provide a better understanding of the framework's capabilities and limitations. Understanding how the framework scales with much larger graphs, potentially with thousands of nodes, is another important area of investigation. This could reveal significant performance advantages over \acrshort{qemc}, but my current computational resources are insufficient to test this effectively at present.
\vspace{5mm}

\noindent\textbf{Real-World Quantum Computer Implementation}: Running the algorithms on an actual quantum computer is a critical step. This will allow us to evaluate how well the framework performs in a real-world quantum environment.
\vspace{5mm}

\noindent\textbf{Further Testing and Refinement}: Lastly, further testing of some of the proposed heuristics is necessary. This includes experimenting with different shot numbers for some of our heuristics to determine their impact on performance. Moreover, there is potential for these heuristics to be refined and improved, leading to even better performance of the framework.
\vspace{5mm}

Concluding, by addressing these areas, future research can build on the foundation laid by this work, potentially leading to significant advancements in the field of Variational Quantum Algorithms (\acrshort{vqa}\textcolor{gray}{s}).

% Just CTRL+F "Future Work" in the main document to find whenever I mention it.

