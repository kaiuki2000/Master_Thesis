%%%%%%%%%%%%%%%%%%%%%%%%%%%%%%%%%%%%%%%%%%%%%%%%%%%%%%%%%%%%%%%%%%%%%%%%
%                                                                      %
%     File: Thesis_Glossary.tex                                        %
%     Tex Master: Thesis.tex                                           %
%                                                                      %
%     Author: Andre C. Marta                                           %
%     Last modified : 27 Feb 2024                                      %
%                                                                      %
%%%%%%%%%%%%%%%%%%%%%%%%%%%%%%%%%%%%%%%%%%%%%%%%%%%%%%%%%%%%%%%%%%%%%%%%
%
% The definitions can be placed anywhere in the document body
% and their order is sorted by <key> automatically when
% calling makeindex in the makefile
%
% ----------------------------------------------------------------------
% To create a glossary entry, use the following syntax:
%
% \newglossaryentry{<label>}{name={<key>}, description={<value>}}
%
% where the parameters are:
% <label> is the label of the entry,
% <key> is the acronym to be defined by the glossary entry (in lowercase, preferably)
% <value> is the actual definition of the current term
%
% To produce the desired term in the document, that will be replaced by
% the user-defined in the output, use the following syntax:
%
% \gls{<label>}
%
% ----------------------------------------------------------------------
% To create a acronym entry, use the following syntax:
%
% \newacronym{⟨label⟩}{⟨abbrv⟩}{⟨full⟩}
%
% where the parameters are:
% <label> is the label of the entry,
% <abbrv> is the acronym,
% <full> is the definition of the acronym
%
% To produce the desired term in the document, that will be replaced by
% the user-defined in the output, use one of the following syntaxes:
%
% \acrlong{<label>}
% \acrshort{<label>}
% \acrfull{<label>}
%
% ----------------------------------------------------------------------
% By default, only those entries defined in the main document using the
% commands above will be displayed in the glossary (list of acronyms),
% unless the command \glsaddall is used,
% ----------------------------------------------------------------------

% The order of the definitions below is irrelevant
% since the glossary is automatically ordered alphabetically

% \newacronym{xdsm}{XDSM}{eXtended Design Structure Matrix}
% \newacronym{csm}{CSM}{Computational Structural Mechanics}
% \newacronym{cfd}{CFD}{Computational Fluid Dynamics}

%\newacronym{⟨label⟩}{⟨abbrv⟩}{⟨full⟩}
%\newacronym{⟨label⟩}{⟨abbrv⟩}{⟨full⟩}
%\newacronym{⟨label⟩}{⟨abbrv⟩}{⟨full⟩}

% Acronyms for QAOA, QEMC, iQAQE, MaxCut, HQCC, VQA, PQC and NISQ.
\newacronym{qaoa}{QAOA}{Quantum Approximate Optimization Algorithm}
\newacronym{qemc}{QEMC}{Qubit-Efficient MaxCut Heuristic (Algorithm)}
\newacronym{iqaqe}{iQAQE}{Interpolated QAOA/QEMC (Hybrid Algorithm)}
\newacronym{maxcut}{MaxCut}{Maximum Cut (Problem)}
\newacronym{hqcc}{HQCC}{Hybrid Quantum-Classical Computing}
\newacronym{vqa}{VQA}{Variational Quantum Algorithm}
\newacronym{pqc}{PQC}{Parameterized Quantum Circuit}
\newacronym{nisq}{NISQ}{Noisy Intermediate-Scale Quantum}
\newacronym{gw}{GW}{Goemans-Williamson (Algorithm)}
\newacronym{ar}{AR}{Approximation Ratio}
\newacronym{tsp}{TSP}{Traveling Salesman Problem}
\newacronym{sat}{SAT}{Boolean Satisfiability Problem}
\newacronym{qkd}{QKD}{Quantum Key Distribution}
\newacronym{vqe}{VQE}{Variational Quantum Eigensolver}
\newacronym{qml}{QML}{Quantum Machine Learning}
\newacronym{sdp}{SDP}{Semidefinite Programming}
\newacronym{hpc}{HPC}{High Performance Computing}
\newacronym{bsf}{BSF}{Best-so-far}
\newacronym{ndcnot}{ND-CNOT}{Non-deterministic CNOT}
\newacronym{dcnot}{D-CNOT}{Deterministic CNOT}
\newacronym{ml}{ML}{Machine Learning}
\newacronym{iv}{IV}{Independent Variable}
\newacronym{dv}{DV}{Dependent Variable}
\newacronym{pca}{PCA}{Principal Component Analysis}
\newacronym{pcr}{PCR}{Principal Component Regression}
\newacronym{vif}{VIF}{Variance Inflation Factor}
\newacronym{pc}{PC}{Principal Component}
\newacronym{svr}{SVR}{Support Vector Machine Regression}
\newacronym{ols}{OLS}{Ordinary Least Squares}

% ----------------------------------------------------------------------
% displays all entries (even those unused with commands \acrlong/short/full)
\glsaddall

% ----------------------------------------------------------------------
% vertical aligment of acronyms' long names
\setlength\LTleft{0pt}
\setlength\LTright{0pt}
\setlength\glsdescwidth{1.0\hsize}
