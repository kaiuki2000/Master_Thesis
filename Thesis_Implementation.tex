%%%%%%%%%%%%%%%%%%%%%%%%%%%%%%%%%%%%%%%%%%%%%%%%%%%%%%%%%%%%%%%%%%%%%%%%
%                                                                      %
%     File: Thesis_Implementation.tex                                  %
%     Tex Master: Thesis.tex                                           %
%                                                                      %
%     Author: Andre C. Marta                                           %
%     Last modified :  4 Mar 2024                                      %
%                                                                      %
%%%%%%%%%%%%%%%%%%%%%%%%%%%%%%%%%%%%%%%%%%%%%%%%%%%%%%%%%%%%%%%%%%%%%%%%

\chapter{Implementation details}
\label{chapter:implementation}

Insert your chapter material here. - In this chapter, we should describe numerical aspects of the algorithms' implementations. This goes for all three of QAOA, QEMC and iQAQE. Mention Pennylane, the developed code/models, the GitHub repository, and any other relevant information. Mention, also, that we're always doing numerical simulations of quantum systems, on a classical computer!

%%%%%%%%%%%%%%%%%%%%%%%%%%%%%%%%%%%%%%%%%%%%%%%%%%%%%%%%%%%%%%%%%%%%%%%%
\section{Individual Algorithms}
\label{section:Individual_Algorithms}

Description of the numerical implementation of the models explained in Chapter~\ref{chapter:Background}.

If needed, pseudo-codes can be included as exemplified in Algorithm~\ref{euclid}.
%
% See package 'algorithmicx' for more information
% https://ctan.org/pkg/algorithmicx
%
\begin{algorithm}
\caption{Euclid’s algorithm}\label{euclid}
\begin{algorithmic}[1]
\Procedure{Euclid}{$a,b$}\Comment{The g.c.d. of a and b}
   \State $r\gets a\bmod b$
   \While{$r\not=0$}\Comment{We have the answer if r is 0}
      \State $a\gets b$
      \State $b\gets r$
      \State $r\gets a\bmod b$
   \EndWhile\label{euclidendwhile}
   \State \textbf{return} $b$\Comment{The gcd is b}
\EndProcedure
\end{algorithmic}
\end{algorithm}

%%%%%%%%%%%%%%%%%%%%%%%%%%%%%%%%%%%%%%%%%%%%%%%%%%%%%%%%%%%%%%%%%%%%%%%%
\subsection{Quantum Approximate Optimization Algorithm (QAOA)}
\label{subsection:QAOA_Implementation}

Description of the numerical implementation of the QAOA model.


%%%%%%%%%%%%%%%%%%%%%%%%%%%%%%%%%%%%%%%%%%%%%%%%%%%%%%%%%%%%%%%%%%%%%%%%
\subsection{Qubit-Efficient MaxCut Heuristic (QEMC)}
\label{subsection:QEMC_Implementation}

Description of the numerical implementation of the QEMC model.


%%%%%%%%%%%%%%%%%%%%%%%%%%%%%%%%%%%%%%%%%%%%%%%%%%%%%%%%%%%%%%%%%%%%%%%%
\subsection{Interpolated QAOA/QEMC Hybrid Algorithm (iQAQE)}
\label{subsection:iQAQE_Implementation}

Description of the numerical implementation of the iQAQE model.

%%%%%%%%%%%%%%%%%%%%%%%%%%%%%%%%%%%%%%%%%%%%%%%%%%%%%%%%%%%%%%%%%%%%%%%%
\section{Benchmarking and Testing methods}
\label{section:Benchmarking_Testing}

\textbf{How do we benchmark our models?} This should be described here: mention the Avg. BSF metric, and how it was "wrong", initially, and how it was "fixed". Also mention any other possible metrics that could be used to compare the models: Grid-searches, etc.

Basic test cases to compare the implemented model against other numerical tools (verification) and experimental data (validation).

I should also introduce the utilized score metrics: Best-so-far average and median, etc. Maybe, mention the difference between the before and after of the "BSF Correction".  

