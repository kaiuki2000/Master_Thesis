\raggedbottom
%%%%%%%%%%%%%%%%%%%%%%%%%%%%%%%%%%%%%%%%%%%%%%%%%%%%%%%%%%%%%%%%%%%%%%%%
%                                                                      %
%     File: Thesis_Introduction.tex                                    %
%     Tex Master: Thesis.tex                                           %
%                                                                      %
%     Author: Andre C. Marta                                           %
%     Last modified :  4 Mar 2024                                      %
%                                                                      %
%%%%%%%%%%%%%%%%%%%%%%%%%%%%%%%%%%%%%%%%%%%%%%%%%%%%%%%%%%%%%%%%%%%%%%%%

\chapter{Introduction}
\label{chapter:introduction}

% Maybe, get some more references for the introduction.

% Insert your chapter material here - Talk briefly about quantum computing and its possible applications in the future. Mention quantum algorithms, hybrid quantum-classical computing, and the \acrshort{maxcut} problem.

Over the last few decades, significant progress has been achieved in the field of quantum computing. This is an entirely new computing paradigm, based on exploiting the fundamental principles of quantum mechanics to our advantage. Although the concept of a quantum computer has been around for a little over 40 years \cite{preskill2023quantum}, only rather recently, in the present century, have we successfully engineered elementary prototypes for such cutting-edge devices. Scientists and engineers worldwide are now working towards the development of practical quantum computers, which are expected to revolutionize the way we solve complex problems in various fields, such as cryptography, optimization, drug discovery, material science, machine learning, and many more.

However, with the current generation of quantum machines, so-called Noisy Intermediate Scale Quantum (\acrshort{nisq}) devices, we are still far from achieving the promised quantum advantage. This refers to the point at which a quantum computer can outperform its classical counterpart by solving specific problems considerably faster or more efficiently. It's the "holy grail" of quantum computing and what drives the research in this field forwards.

Presently, the most promising candidates for achieving meaningful quantum advantage are anchored in what has become known as "Hybrid Quantum-Classical Computing" (\acrshort{hqcc}). This approach aims to merge the strengths of both worlds: the computational power of quantum and the stability of classical computing. In this context, Variational Quantum Algorithms (\acrshort{vqa}\textcolor{gray}{s}) have been at the forefront of research, as they are particularly well-suited for near-term quantum devices, requiring fewer qubits and featuring shallower circuit depths. These algorithms are hybrid by design, using a classical optimizer to adjust the parameters of a Parameterized Quantum Circuit (\acrshort{pqc}).

Amid the many prospective applications of quantum computing, notable advances have been made in recent years in solving combinatorial optimization problems. \acrshort{vqa}\textcolor{gray}{s} like the Quantum Approximate Optimization Algorithm (\acrshort{qaoa}) \cite{farhi2014quantum} have demonstrated potential for tackling the Maximum Cut (\acrshort{maxcut}) problem, a challenging graph-based NP-hard problem in computer science. Despite its promise, however, \acrshort{qaoa} has a significant drawback: it requires a large number of qubits, exceeding the capacity of current quantum devices, when scaled to larger problem instances. This constraint has prompted the development of alternative approaches such as the Qubit-Efficient \acrshort{maxcut} Heuristic Algorithm (\acrshort{qemc}) \cite{tenecohen2023variational}, designed to address the \acrshort{maxcut} problem using fewer qubits. Meanwhile, the search for better algorithms to solve this problem continues, and the development of new hybrid quantum-classical methods is crucial to achieving quantum advantage.

%%%%%%%%%%%%%%%%%%%%%%%%%%%%%%%%%%%%%%%%%%%%%%%%%%%%%%%%%%%%%%%%%%%%%%%%
\section{Motivation}
\label{section:motivation}

% \textbf{Why do we care?}

% Relevance of the subject. - Explain why the topic is important and why it is worth studying. What would be the impact of solving the \acrshort{maxcut} problem in a more efficient manner? What are the potential applications of the results?

The search for efficient algorithms to solve combinatorial optimization problems is critical in numerous fields, such as logistics, finance, and telecommunications. The \acrshort{maxcut} problem, for example, has broad applications in areas like machine learning \cite{937505}, statistical physics \cite{Barahona_Grötschel_Jünger_Reinelt_1988}, circuit design \cite{Barahona_Grötschel_Jünger_Reinelt_1988}, and data clustering \cite{10.1007/11893318_21}. Creating more efficient algorithms to solve this problem can improve the performance of these applications, driving substantial progress in their respective fields.

Moreover, there is a strong interest from the computer science community in terms of computational complexity. The \acrshort{maxcut} problem is recognized as NP-hard, and the search for efficient algorithms to solve it may offer valuable insights into the boundaries between classical and quantum computing. It might even contribute to unraveling one of the most perplexing questions in theoretical computer science: the $P = NP$ problem. Imagining a world where the $P = NP$ conjecture is proven true, albeit improbable, is intriguing. It would mean that every problem in $NP$ could be solved in polynomial time, including \acrshort{maxcut}. This would also extend to problems like the Traveling Salesman Problem (\acrshort{tsp}) and the Knapsack Problem, among others. The implications would be profound, as this would dramatically increase our ability to solve previously difficult optimization problems, with direct applications in areas like vehicle routing, job scheduling, and broader logistics. Additionally, the impact on cryptography would be as significant – if not more so – since many cryptographic techniques rely on the complexity of $NP$ problems. For example, integer factorization is a key component of RSA (Rivest-Shamir-Adleman) encryption, a popular asymmetric encryption algorithm used extensively in public-key cryptography for secure message transmission over the internet. If $P = NP$, RSA encryption could easily be broken, leading to major security risks. Hence, the importance of studying these problems to fully understand their complexity.

The aforementioned considerations drive our efforts to develop a new algorithm for solving the \acrshort{maxcut} problem with greater efficiency and accuracy. This algorithm, the Interpolated QAOA/QEMC (\acrshort{iqaqe}) Hybrid Algorithm, will be the focus of this thesis. More specifically, we will develop what we call the \acrshort{iqaqe} Framework, which will serve as the foundation for developing numerous distinct \acrshort{iqaqe} algorithms.

% Re-read this one more time, I think. I believe I end up repeating myself a bit, when I mention applications, again!

%%%%%%%%%%%%%%%%%%%%%%%%%%%%%%%%%%%%%%%%%%%%%%%%%%%%%%%%%%%%%%%%%%%%%%%%
\section{Topic Overview}
\label{section:overview}

% Briefly, dumbly, describe the topic. - Explain what variational quantum algorithms are, what QAOA and QEMC are, and what the \acrshort{maxcut} problem is. Mention the importance of hybrid quantum-classical computing and the potential of combining these two algorithms. Describe iQAQE, and how it might help in achieving the objectives.

% \textbf{What are we going to be working on?}

% Provide an overview of the topic to be studied. - Briefly describe variational quantum algorithms, QAOA and QEMC, and the \acrshort{maxcut} problem. Explain the importance of hybrid quantum-classical computing and the potential of combining these two algorithms.

In this project, we propose the Interpolated QAOA/QEMC (\acrshort{iqaqe}) Framework. This will enable us to develop novel \acrshort{vqa}\textcolor{gray}{s} by leveraging the strengths of two existing \acrshort{vqa}\textcolor{gray}{s}, \acrshort{qaoa} and \acrshort{qemc}, for improved performance in solving the \acrshort{maxcut} problem. \acrshort{qaoa} requires a qubit for each graph vertex, making it difficult to scale. In contrast, \acrshort{qemc} uses exponentially fewer qubits by assigning one basis state to each graph node, requiring only $\log_2(n)$ qubits (for $n$ graph vertices). However, this compression leads to limitations in \acrshort{qemc}'s results. By interpolating both \acrshort{vqa}\textcolor{gray}{s}, we aim to create an algorithm that utilizes fewer qubits than \acrshort{qaoa} and performs better than \acrshort{qaoa} and \acrshort{qemc}. The new algorithm, constructed through the \acrshort{iqaqe} Framework, assigns multiple basis states to each node, in contrast to \acrshort{qemc}'s single basis state approach. This design tentatively allows for a more practical implementation on present-day \acrshort{nisq} devices, thanks to its reduced qubit requirements compared to \acrshort{qaoa}, fewer measurement shots than \acrshort{qemc}, and potentially greater trainability than \acrshort{qaoa}.

%%%%%%%%%%%%%%%%%%%%%%%%%%%%%%%%%%%%%%%%%%%%%%%%%%%%%%%%%%%%%%%%%%%%%%%%
\section{Objectives} % Changed from "Objectives and Deliverables" to just "Objectives" to simplify the title.
\label{section:objectives}

% Should be short. The objectives are, essentially, the same as the motivation. 'Deliverables' can either be removed, or just mention the algorithm's code (iQAQE) and the expected results.

% \textbf{What are our goals? What are we going to deliver?}

% Explicitly state the objectives set to be achieved with this thesis. - What are the main goals of the work? Mention iQAQE and how it might help in achieving the objectives.

% Also list the expected deliverables. - Results (plots, tables, etc.) and code.

The primary objective of this thesis is to develop and analyze the \acrshort{iqaqe} Framework. Algorithms built within this framework will be implemented and tested using classical simulations of quantum machines. The deliverables for this project include the \acrshort{iqaqe} Framework's code\footnote{Reach out to the author to obtain access to the code.}, the results obtained from the simulations, and the analysis of these results. The expected outcomes are improvements in the performance of the \acrshort{iqaqe} algorithm compared to \acrshort{qaoa} and \acrshort{qemc}, with a focus on accuracy, efficiency, and scalability.

%%%%%%%%%%%%%%%%%%%%%%%%%%%%%%%%%%%%%%%%%%%%%%%%%%%%%%%%%%%%%%%%%%%%%%%%
\section{Thesis Outline}
\label{section:outline}

% This one is easy. Just mention the structure of the thesis.

% \textbf{What does this work's/thesis's structure look like?}

% Briefly explain the contents of each chapter.

% Yeahhh, re-read this again. I've changed some things in between the first time I wrote this and now.
This thesis is structured as follows: Chapter~\ref{chapter:Background} provides an overview of the background concepts related to quantum computing, variational quantum algorithms, and the \acrshort{maxcut} problem. Chapter~\ref{chapter:Base Algorithm} introduces the hybrid quantum-classical \acrshort{iqaqe} Framework, explaining its design, advantages, and potential applications. Chapter~\ref{chapter:implementation} details the numerical implementation of the \acrshort{iqaqe} Framework and individual algorithms \acrshort{qaoa} and \acrshort{qemc}, in addition to the Goemans-Williamson (\acrshort{gw}) algorithm. It also describes the benchmarking and testing methods used to evaluate the algorithms' performances. In Chapter~\ref{chapter:Schemes_and_Results}, we delve into the different \acrshort{iqaqe} schemes developed and the corresponding results obtained from simulations. These are compared with \acrshort{qaoa}, \acrshort{qemc}, and \acrshort{gw}. Chapter~\ref{chapter:Exploratory_Ideas} discusses alternative approaches to solving the \acrshort{maxcut} problem, derived from our research on \acrshort{iqaqe}. Finally, Chapter~\ref{chapter:conclusions} concludes the thesis, summarizing the main findings and suggesting future research directions.

% I've re-read this once! I think it's fine (for now).